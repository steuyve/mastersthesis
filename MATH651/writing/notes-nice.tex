\documentclass[11pt]{article}

\usepackage{hyperref}
\usepackage{mathtools}
\usepackage{amsthm}
\usepackage{amssymb}
\usepackage{MnSymbol}
\usepackage{mathrsfs}
\usepackage[arrow]{xy}
\usepackage{dsfont}
\usepackage{enumitem}
\usepackage{accents}
\usepackage{tabu}

\pagestyle{headings}

\newcommand{\Z}{\mathbb{Z}}

\theoremstyle{definition}
\newtheorem*{defn}{Definition}

\theoremstyle{definition}
\newtheorem{ex}{Example}

\theoremstyle{plain}
\newtheorem{theo}{Theorem}

\theoremstyle{plain}
\newtheorem*{prop}{Proposition}

\theoremstyle{plain}
\newtheorem{lem}{Lemma}

\theoremstyle{definition}
\newtheorem{que}{Question}

\begin{document}

\author{Stephen Liu}
\title{Thesis Notes}
\date{September 4, 2019}

\maketitle

\begin{abstract}
\end{abstract}

\tableofcontents
\newpage

\section{Introduction}

\section{Finite metric spaces}

\subsection{The magnitude of a finite metric space}

\subsection{The magnitude function}

\subsection{Finite positive definite metric spaces}

\section{Infinite metric spaces}

\section{Odd dimensional Euclidean balls}

\section{The problem}

Let $d = 2m+1$ where $m$ is a natural number. Denote by $B_2^d$ the $d$-dimensional Euclidean ball and $tB_2^d$ the $d$-dimensional ball with radius $t$ for $t > 0$. In general we are interested in the asymptotic behavior of the magnitude function of the odd dimensional Euclidean ball as we shrink it down to a point. Meckes in \cite{meckes_magnitude_2019} showed that
\begin{equation*}
\frac{d}{dt}\text{Mag}(tB_2^d)\big\vert_{t=0} = \frac{1}{2}V_1(B_2^d)
\end{equation*}
where $V_1(B_2^d)$ is the first instrinsic volume of the Euclidean ball.
We are interested computing the value of
\begin{equation*}
\frac{d^2\text{Mag}(tB_2^d)}{dt^2}\big\vert_{t=0}.
\end{equation*}
We will follow the same approach that Meckes used for the result above, with many of the techniques introduced by Willerton in \cite{willerton_magnitude_2017}.

\subsection{Explicit magnitude functions for balls in low dimension}

Barcelo and Carbery in \cite{barcelo_magnitudes_2016} explicitly calculated the magnitude functions of Euclidean balls in dimensions 3,5,7:
\begin{align*}
&\text{Mag}(tB_2^3) = \frac{t^3}{3!}+t^2+2t+1 \\
&\text{Mag}(tB_2^5) = \frac{t^6+18t^5+135t^4+525t^3+1080t^2+1080t+360}{5!(t+3)} \\
&\text{Mag}(tB_2^7) = \frac{t^7}{7!} \\
&+ \frac{\frac{1}{180}t^9+\frac{2}{15}t^8+\frac{3}{2}t^7+\frac{31}{3}t^6+\frac{189}{4}t^5+145t^4+\frac{1165}{4}t^3+360t^2+240t+60}{t^3+12t^2+48t+60}
\end{align*}
So taking second derivatives and evaluating at $t = 0$, we have
\begin{align*}
&\frac{d\text{Mag}^2(tB_2^3)}{dt^2}\big\vert_{t=0} = 2 \\
&\frac{d\text{Mag}^2(tB_2^5)}{dt^2}\big\vert_{t=0} = \frac{38}{9} = 4.222\dots \\
&\frac{d\text{Mag}^2(tB_2^7)}{dt^2}\big\vert_{t=0} = \frac{162}{25} = 6.48 \\
\end{align*}

We will use these values as sanity checks to compare against throughout our calculations.

\section{Schröder paths}

Willerton in \cite{willerton_magnitude_2017} gave an expression for the magnitude function of odd dimensional balls in terms of collections of Schröder paths. We introduce them below:
\begin{defn}
\begin{enumerate}[label=$\bullet$]
\item A \textbf{Schröder path} is a finite directed path in the integer lattice in which each step $(x,y)\in\Z^2$ is either an \textbf{ascent} to $(x+1,y+1)$, a \textbf{descent} $(x-1,y-1)$ or a \textbf{flat step} $(x+2,y)$ (note the advance by \emph{two} spaces in the horizontal direction).
\item Fix $k\geq0$. A \textbf{disjoint k-collection} is a family of Schröder paths from $(-i,i)$ to $(i,i)$ for each $0\leq i\leq k$ such that no node in $\Z^2$ is contained in two of the paths (the paths are disjoint).
\item We denote by $X_k$ the set of all disjoint $k$-collections and by $X_k^j$ the set of disjoint $k$-collections with exactly $j$ flat steps.
\end{enumerate}
\end{defn}

When thinking about what kinds of disjoint $k$-collections we can have in $X_k^j$ for some fixed $j$, it is often useful to think about what a path needs to look like for increasing values of $i$. For example, consider the set $X_k^0$, that is, the set of disjoint $k$-collections with exactly 0 flat steps. At $i = 0$ we have the single dot at $(0,0)$ and at $i = 1$, since we are allowed no flat steps, the only possible path we can have is the path made up of one ascent followed immediately by one descent. Then for $i = 2$, the disjointness condition and the presence of the earlier path at height $i=1$ ensures that the only possible path we can have is the path made up of two ascents followed by two descents. We continue this argument for successive values of $i$. We will call the path at height $i$ consisting of $i$ ascents followed by $i$ descents a \textbf{V-path at height $i$} (because they look like upside down V's). So it turns out that $X_k^0$ consists only of one collection, denoted $\sigma_{\text{roof}}^k$, which is made up entirely of V-paths for each $0\leq i\leq k$.

Let $\sigma$ be a disjoint $k$-collection in $X_k$. For each path in $\sigma$ we associate a weighting to each step $\tau$ in the path by the following:
\begin{equation*}
\omega_j(\tau) = \begin{cases} 
1 &\text{if $\tau$ is an ascent,} \\
t &\text{if $\tau$ is a flat step,} \\
y+1-j &\text{if $\tau$ is a descent from height $y$ to height $y-1$.}
\end{cases}
\end{equation*}
For a collection $\sigma\in X_k$ the \textbf{(total) weight of $\sigma$}, denoted by $\omega_j(\sigma)$ is the product of all the weightings of each step of a path in $\sigma$, that is,
\begin{equation*}
\omega_j(\sigma) = \prod\limits_{\tau\in\sigma}\omega_j(\tau)
\end{equation*}
Note that if $\sigma \in X_k^\ell$, that is $\sigma$ has exactly $\ell$ flat steps, then $\omega_j(\sigma)$ will take the form $ct^\ell$ where $c$ is the product of all the weights on descents in $\sigma$. We will only use the weightings $\omega_0$ and $\omega_2$ for our purposes, that is, a descent starting at height $y$ will be weighted by either $y+1$ and $y-1$ respectively. Consider a V-path $\sigma$ at height $i$, then we have
\begin{align*}
&\omega_0(\sigma) = \frac{(2i+1)!}{(i+1)!} \\
&\omega_2(\sigma) = \frac{(2i-1)!}{(i-1)!}
\end{align*}
and since $\sigma_{\text{roof}}^k$ consists only of V-paths, we have
\begin{align*}
&\omega_0\left(\sigma_{\text{roof}}^k\right) = \prod\limits_{i=0}^k\frac{(2i+1)!}{(i+1)!} \\
&\omega_2\left(\sigma_{\text{roof}}^k\right) = \prod\limits_{i=1}^k\frac{(2i-1)!}{(i-1)!} \\
\end{align*}

We are interested in these weightings on collections of Schröder paths because Willerton in \cite{willerton_magnitude_2017} showed the following:

\begin{theo}[\cite{willerton_magnitude_2017}, Corollary 27]
Let $d = 2m+1$ be odd. Then
\begin{equation*}
\text{Mag}\left(tB_2^d\right) = \frac{\sum\limits_{\sigma\in X_{m+1}}\omega_2(\sigma)}{d!\sum\limits_{\sigma\in X_{m-1}}\omega_0(\sigma)} = \frac{\sum\limits_{\sigma\in X_{m+1}}\prod\limits_{\tau\in\sigma}\omega_2(\tau)}{d!\sum\limits_{\sigma\in X_{m-1}}\prod\limits_{\tau\in\sigma}\omega_0(\tau)}
\end{equation*}
for all $t > 0$.
\end{theo}
As mentioned before, $\omega_j(\sigma)$ are of the form $ct^\ell$ where $\ell$ is the number of flat steps in $\sigma$ and $c$ is a constant, so the numerator and the denominator in the expression above are both polynomials in $t$. We will denote the function in the numerator by $N(t)$ and the function in the denominator (without the extra $d!$) by $D(t)$. Put more succinctly, we have
\begin{equation*}
\text{Mag}\left(tB_2^d\right) = \frac{N(t)}{d!D(t)}.
\end{equation*}

\section{Taking the second derivative}

From now on, when writing down the value of a function evaluated at zero, for convenience we will omit the ``$(0)$'' part, that is, we write $N$ for $N(0)$ and $N'$ for $N'(0)$ and similarly for $D(0)$ and $D'(0)$. For higher derivatives we divide by the order of the derivative, that is, we denote $\frac{1}{2}N''(0)$ by $N''$ and $\frac{1}{2}D''(0)$ by $D''$. The point of this is that $N''$ and $D''$ are the values of the coefficients of the second order terms in $N(t)$ and $D(t)$ respectively. In Theorem 28 of \cite{willerton_magnitude_2017}, Willerton showed the following identity
\begin{equation*}
N = d!D
\end{equation*}
and Meckes in the proof of Theorem 4 of \cite{meckes_magnitude_2019} showed that
\begin{equation*}
N'D - ND' = \frac{1}{2}V_1d!D^2
\end{equation*}
where $V_1$ is shorthand for $V_1\left(B_2^d\right)$. Now we evaluate
\begin{equation*}
\frac{d^2}{dt^2}\text{Mag}(tB_2^d)\big\vert_{t=0}
\end{equation*}
Applying the quotient rule and using the two identities above, we have
\begin{align*}
\frac{d^2}{dt^2}\text{Mag}(tB_2^d)\big\vert_{t=0} &= \frac{d}{dt}\left(\frac{d}{dt}\text{Mag}(tB_2^d)\right)\big\vert_{t=0} \\
&= \frac{d}{dt}\left(\frac{d}{dt}\frac{N(t)}{d!D(t)}\right)\big\vert_{t=0} \\
&= \frac{d}{dt}\left(\frac{d!D(t)N'(t)-N(t)d!D'(t)}{d!^2D(t)^2}\right)\big\vert_{t=0} \\
&= \frac{d}{dt}\left(\frac{D(t)N'(t)-N(t)D'(t)}{d!D(t)^2}\right)\big\vert_{t=0} \\
&= \frac{d!D(t)^2[D(t)N''(t)-N(t)D''(t)]-[D(t)N'(t)-N(t)D'(t)]d!2D(t)D'(t)}{d!^2D(t)^4}\big\vert_{t=0} \\
&= \frac{D(t)[D(t)N''(t)-N(t)D''(t)]-[D(t)N'(t)-N(t)D'(t)]2D'(t)}{d!D(t)^3}\big\vert_{t=0} \\
&= \frac{D(t)^2N''(t)-D(t)N(t)D''(t)-2D'(t)D(t)N'(t)+2D'(t)^2N(t)}{d!D(t)^3}\big\vert_{t=0} \\
&= \frac{2D^2N''-2DD'N'-2DND''+2ND'^2}{d!D^3} \\
&= \frac{2D^2N''-2DND''-2D'(DN'-D'N)}{d!D^3} \\
&= \frac{2D^2N''-2DND''-2D'(\frac{1}{2}V_1d!D^2)}{d!D^3} \\
&= 2\left[\frac{DN''-ND''}{d!D^2}\right]-V_1\left[\frac{D'}{D}\right] \\
&= 2\left[\frac{DN''-d!DD''}{d!D^2}\right] - V_1\left[\frac{D'}{D}\right] \\
&= 2\left[\frac{N''-d!D''}{d!D}\right] - V_1\left[\frac{D'}{D}\right] \\
&= 2\left[\frac{N''-d!D''}{N}\right] - V_1\left[\frac{D'}{D}\right]
\end{align*}
Where the factors of 2 in front of the terms containing a second derivative are because of the factor of $\frac{1}{2}$ that we introduced earlier in our notation for $N''$ and $D''$. To continue simplifying this expression, we have to give explicit expressions for the terms involved:
\begin{align*}
&N = \sum\limits_{\sigma\in X_{m+1}^0}\prod\limits_{\tau\in\sigma}\omega_2(\tau) = \prod\limits_{\tau\in\sigma_{\text{roof}}^{m+1}}\omega_2(\tau),\quad D = \sum\limits_{\sigma\in X_{m-1}^0}\prod\limits_{\tau\in\sigma}\omega_0(\tau) = \prod\limits_{\tau\in\sigma_{roof}^{m-1}}\omega_0(\tau) \\
&N' = t^{-1}\sum\limits_{\sigma\in X_{m+1}^1}\prod\limits_{\tau\in\sigma}\omega_2(\tau),\quad D' = t^{-1}\sum\limits_{\sigma\in X_{m-1}^1}\prod\limits_{\tau\in\sigma}\omega_0(\tau) \\
&N'' = t^{-2}\sum\limits_{\sigma\in X_{m+1}^2}\prod\limits_{\tau\in\sigma}\omega_2(\tau),\quad D'' = t^{-2}\sum\limits_{\sigma\in X_{m-1}^2}\prod\limits_{\tau\in\sigma}\omega_0(\tau)
\end{align*}
This boils down to a counting problem to do with disjoint collections of Schröder paths with exactly $k$ flat steps for $k = 0,1,2$.

\subsection{Simplifying the $\frac{D'}{D}$ Term}

By above we have,
\begin{equation*}
D = \prod\limits_{\tau\in\sigma_{roof}^{m-1}}\omega_0(\tau) = \prod\limits_{k=0}^{m-1}\frac{(2k+1)!}{(k+1)!}
\end{equation*}
and
\begin{equation*}
D' = t^{-1}\sum\limits_{\sigma\in X_{m-1}^1}\prod\limits_{\tau\in\sigma}\omega_0(\tau)
\end{equation*}
so we want to count how many disjoint $(m-1)$-collections $\sigma$ have exactly one flat step in them.

Suppose we have a disjoint $k$-collection containing exactly one flat step. Let $\sigma$ be the path in this collection containing the single flat step at height, say, $p$. By the same reasoning as when discussing $X_k^0$ earlier, the paths at height less than $p$ must be V-paths. But then the disjointness condition ensures that $\sigma$ at height $p$ must have the flat step be centered, so $\sigma$ is a path consisting of $p-1$ ascents, one flat step and then $p-1$ descents. We will call such a path a \textbf{flat step path at height $p$}. Note that the weightings for this kind of path are given by
\begin{align*}
&\omega_0(\sigma) = \frac{(2p)!}{(p+1)!} \\
&\omega_2(\sigma) = \frac{(2p-1)!}{(p-1)!}
\end{align*}
For the path directly above $\sigma$, we can either have a V-path as before or, because of the extra space provided by the flat step just below we can have a path consisting of $p$ ascents, one descent, one ascent and then $p$ descents. Then for the next path above, the disjointness condition ensures that this path can only be either a path of a similar form or a V-path. We will call the path at height $k$ consisting of $k-1$ ascents, one descent, one ascent and then $k$ descents a \textbf{M-path at height $k$}. So in $\sigma$, after the flat step path at height $p$ we will have some number of M-paths followed by some number of V-paths. Note that after we have switched to V-paths we cannot have any other paths above because of the disjointness condition. This allows us to characterize all the disjoint $(m-1)$-collections in $X_{m-1}^1$: fix $1\leq p\leq m-1$ and $0\leq q \leq m-1-p$, then the disjoint $(m-1)$-collection $\sigma_{p,q}^{m-1}$ from the bottom up, is composed of $p-1$ V-paths, followed by a flat step path at height $p$, followed by $q$ M-paths, followed by V-paths up to height $m-1$. Then Meckes in \cite{meckes_magnitude_2019} observed that
\begin{equation*}
X_{m-1}^1 = \bigcup\limits_{\substack{1\leq p \leq m-1 \\ 0\leq q \leq m-1-p}} \sigma_{p,q}^{m-1}
\end{equation*}

Let $\sigma$ be a M-path at height $k$, then we have
\begin{align*}
&\omega_0(\sigma) = \frac{(2k)!(2k)}{(k+1)!} \\
&\omega_2(\sigma) = \frac{(2k-1)!(2k-1)}{(k-1)!}
\end{align*}
where the extra factor on the numerator comes from the additional descent we have in $\sigma$. The $\omega_2$ weighting will become relevant later on when we evaluate $N$.

Since we can recognize $X_{m-1}^1$ as a union of disjoint $(m-1)$-collections with this specific form, we can write down an explicit expression for $D'$:
\begin{equation*}
D' = \sum\limits_{\substack{1\leq p\leq m-1\\0\leq q\leq m-1-p}}\left(\prod\limits_{k=0}^{p-1}\frac{(2k+1)!}{(k+1)!}\right)\left(\frac{(2p)!}{(p+1)!}\right)\left(\prod\limits_{k=p+1}^{p+q}\frac{(2k)!(2k)}{(k+1)!}\right)\left(\prod\limits_{k=p+q+1}^{m-1}\frac{(2k+1)!}{(k+1)!}\right)
\end{equation*}

We can simplify the quotient $D'/D$: For each summand depending on $p,q$ in the quotient $D'/D$, we have
\begin{equation*}
\frac{\prod\limits_{k=0}^{p-1}\frac{1}{(k+1)!}\frac{1}{p!}\prod\limits_{k=p+1}^{p+q}\frac{1}{(k+1)!}\prod\limits_{k=p+q+1}^{m-1}\frac{1}{(k+1)!}\prod\limits_{k=0}^{p-1}(2k+1)!(2p)!\prod\limits_{k=p+1}^{p+q}(2k)!(2k)\prod\limits_{k=p+q+1}^{m-1}(2k+1)!}{\prod\limits_{k=0}^{m-1}\frac{1}{(k+1)!}\prod\limits_{k=0}^{m-1}(2k+1)!}
\end{equation*}
We can cancel the product of all the $\frac{1}{(k+1)!}$'s since on the top we also have a product of $\frac{1}{(k+1)!}$'s from 0 up to $m-1$. This gives us
\begin{equation*}
\frac{\prod\limits_{k=0}^{p-1}(2k+1)!(2p)!\prod\limits_{k=p+1}^{p+q}(2k)!(2k)\prod\limits_{k=p+q+1}^{m-1}(2k+1)!}{\prod\limits_{k=0}^{m-1}(2k+1)!}
\end{equation*}
We can further cancel all the $(2k+1)!$'s from $k = 0$ to $p-1$ and from $p+q+1$ to $m-1$:
\begin{align*}
\frac{(2p)!\prod\limits_{k=p+1}^{p+q}(2k)!(2k)}{\prod\limits_{k=p}^{p+q}(2k+1)!} &= \frac{(2p)!\prod\limits_{k=p+1}^{p+q}(2k)!(2k)}{(2p+1)!\prod\limits_{k=p+1}^{p+q}(2k+1)! }\\
&=\frac{1}{2p+1}\left(\prod\limits_{k=p+1}^{p+q}\frac{2k}{2k+1}\right) \\
\end{align*}
So summing over all such $p$ and $q$ we have
\begin{equation*}
\frac{D'}{D} = \sum\limits_{\substack{1\leq p \leq m-1 \\ 0 \leq q \leq m - 1 - p}}\frac{1}{2p+1}\prod\limits_{k=p+1}^{p+q}\left(\frac{2k}{2k+1}\right)
\end{equation*}

\subsection{Disjoint $k$-collections containing exactly two flat steps}

Recall for $N''$ and $D''$ we are considering paths in disjoint $k$-collections from either $X_{m+1}^2$ or $X_{m-1}^2$. So our next step is to describe all disjoint $k$-collections containing exactly two flat steps. We can already rule out the possibility where a disjoint $k$-collection has two flat steps on the same path. This is because any path below the one with the flat steps needs to be a V-path and there's then not enough room on the path with the flat step for a flat step to appear anywhere other than the center. This then also tells us that for any disjoint $k$-collection with two flat steps, the two flat steps will be on separate paths and moreover the first flat step will be centered.

As before, after the first flat step we can have some number of M-paths followed by some number of V-paths. If we have a nonzero amount of V-paths, then that means the second path with a flat step in it will have its flat step centered. On the other hand, if we only have M-paths above the first flat step path, then we have enough room for the second path containing a flat step to have its flat step offset by one either to the left or right, that is, either a path starting at height $k$ with $(k-2)$ ascents, followed by a flat step, one more ascent and then $(k-1)$ descents, or a path starting at height $k$ with $(k-1)$ ascents, one descent, a flat step and then $(k-2)$ descents. Notice that by symmetry, a path with flat step offset to the left has same total weight as a path with flat step offset to the right, and moreover that both have a total weight  equal to the total weight of a regular flat step path at the same height. For this reason, in the future when discussing the total weight, we will use the term ``flat step path'' to refer to both types of paths. Suppose we have a left-offset flat step path at height $p$, then on the path at height $p+1$ we have enough space to have a left \textbf{asymmetric M-path} consisting of $(p+1-2)$ ascents, followed by one descent, 2 ascents, and then $(p+1-1)$ descents. More generally, a (left) asymmetric M-path at height $k$ consists of $(k-2)$ ascents, one descent, two ascents and then $(k-1)$ descents. Above this first left asymmetric M-path we can have more asymmetric M-paths or regular M-paths or V-paths.

Suppose we have an left asymmetric M-path $\sigma$ at height $k$, then the product of the weights on this path is given by
\begin{equation*}
\omega_2(\sigma) = \frac{(2k-2)!}{(k-1)!}(2k-3)
\end{equation*}
Notice that by symmetry, a right asymmetric M-path (corresponding to the case where the second flat step is off-set to the right) will have the same product of weights. This means that a disjoint $k$-collection of where the second flat step is offset to the left followed by left asymmetric M-paths will have the same total weighting as the disjoint $k$-collection yielded by reflecting across the $y$-axis. For this reason, we will only need to consider the left offset case.

To summarize, we have two cases of disjoint $k$-collections containing exactly two flat steps:
\begin{enumerate}[label=(\alph*)]
\item Two centred flat steps: We have two centered flat step paths at height $p_1$ and $p_2$ respectively. Above the first flat step path we have $q_1$ M-paths and above the second flat step path we have $q_2$ M-paths. V-paths fill in all the rest. We will denote these kinds of disjoint $k$-collections by $\sigma_{p_1,p_2,q_1,q_2}^k$.
\item Second flat step is offset: We have one centered flat step path at height $p_1$ and an offset flat step path at height $p_2$. In between $p_1$ and $p_2$ we have only M-paths. Above the second flat step path we have $q_1$ asymmetric M-paths followed by $q_2$ M-paths. V-paths fill in the top and the bottom. We will denote disjoint $k$-collections of this form by $L_{p_1,p_2,q_1,q_2}^k$ for a left offset and $R_{p_1,p_2,q_1,q_2}^k$ for a right offset (though as remarked above, we will only need to consider the left offset case).
\end{enumerate}

So we have that
\begin{equation*}
X_{k}^2 = \bigcup\limits_{p_1,p_2,q_1,q_2}\sigma_{p_1,p_2,q_1,q_2}^k \cup \bigcup\limits_{p_1,p_2,q_1,q_2}L_{p_1,p_2,q_1,q_2}^k \cup \bigcup\limits_{p_1,p_2,q_1,q_2}R_{p_1,p_2,q_1,q_2}^k
\end{equation*}

\subsection{$\mu$ Simplification of the $N''-d!D''$ Term}

Recall that for $N''$ and $D''$ we have:
\begin{equation*}
N'' = t^{-2}\sum\limits_{\sigma\in X_{m+1}^2}\prod\limits_{\tau\in\sigma}\omega_2(\tau),\quad D'' = t^{-2}\sum\limits_{\sigma\in X_{m-1}^2}\prod\limits_{\tau\in\sigma}\omega_0(\tau)
\end{equation*}
so in $N''$ we are considering disjoint $(m+1)$-collections with exactly two flat steps while in $D''$ we are considering disjoint $(m-1)$-collections. In order to not have to consider both, we employ the same trick originally used in \cite{willerton_magnitude_2017} to show $N = d!D$ (which we will call $\mu$ simplification). The idea is to view disjoint $(m-1)$-collections in $X_{m-1}$ as being embedded in $X_{m+1}$ and so we only need to work in $X_{m+1}^2$. Let $\sigma \in X_{m-1}$, then we get a corresponding $\mu(\sigma)\in X_{m+1}$ by shifting all paths up two units, adding ascents from $(-i,i)$ to $(-i+1,i+1)$ and descents from $(i-1,i+1)$ to $(i,i)$ for $1\leq i \leq m$, and finally adding a V-path at height $m+1$. Then $\mu(\sigma)$ has the same number of flat steps as $\sigma$ and
\begin{equation*}
\prod\limits_{\tau\in\mu(\sigma)}\omega_2(\tau) = d!\prod\limits_{\tau\in\sigma}\omega_0(\tau)
\end{equation*}
Let $\mu(X_{m-1}^2)\subseteq X_{m+1}^2$ denote the disjoint $(m+1)$-collections that are these embeddings of all the disjoint $(m-1)$-collections in $X_{m-1}^2$, then by the relation above, we have
\begin{align*}
N''-d!D'' &= t^{-2}\sum\limits_{\sigma\in X_{m+1}^2}\prod\limits_{\tau\in\sigma}\omega_2(\tau) - d!t^{-2}\sum\limits_{\sigma\in X_{m-1}^2}\prod\limits_{\tau\in\sigma}\omega_0(\tau) \\
&= t^{-2}\sum\limits_{\sigma\in X_{m+1}^2\setminus\mu(X_{m-1}^2)}\prod\limits_{\tau\in\sigma}\omega_2(\tau)
\end{align*}
So our next step is to describe all disjoint $(m+1)$-collections with two flat steps that are not embeddings of disjoint $(m-1)$-collections with two flat steps. We have four disjoint cases:
\begin{enumerate}
\item The first flat step at height $p_1=1$ and the second flat step is at height $p_2$ where $2\leq p_2\leq m+1$.
\item The first flat step is at height $p_1 \geq 2$. The second flat step is at height $p_2 = m$ and we either have a M-path or an asymmetric M-path above $p_2$.
\item The first flat step is at height $p_1 \geq 2$. The second flat step is at height $p_2 = m+1$.
\item The two flat steps are at heights between $2$ and $m-1$ but with no V-paths above height $p_2$.
\end{enumerate}
Abusing notation slightly, let $\sigma$ be the total weighting of the disjoint collections that are in any of the four cases above and moreover have two centered flat steps. Let $L$ be the corresponding weighting for disjoint collections where the second flat step is offset to the left and let $R$ be the corresponding weighting where the second flat step is offset to the right. By the symmetry reasoning above, we have that $L=R$, so we have that
\begin{equation*}
N'' - d!D'' = \sigma + L + R = \sigma + 2L
\end{equation*}
We explicitly give expressions for $\sigma$ and $L$ below:
\begin{align*}
\sigma = &\sum\limits_{\substack{2\leq p_2\leq m+1 \\ 0\leq q_1 \leq p_2-2 \\ 0 \leq q_2 \leq m+1-p_2}}\omega_2(\sigma_{1,p_2,q_1,q_2}^{m+1}) + \sum\limits_{\substack{2 \leq p_1 \leq m-1 \\ 0 \leq q_1 \leq m-p_1-1}}\omega_2(\sigma_{p_1,m,q_1,m+1-p_2}^{m+1}) \\
 &\quad+ \sum\limits_{\substack{2 \leq p_1 \leq m \\ 0 \leq q_1 \leq m-p_1}}\omega_2(\sigma_{p_1,m+1,q_1,0}^{m+1}) + \sum\limits_{\substack{2 \leq p_1 \leq m \\ p_1+1 \leq p_2 \leq m-1 \\ 0 \leq q_1 \leq p_2-p_1-1}} \omega_2(\sigma_{p_1,p_2,q_1,m+1-p_2}^{m+1})
\end{align*}
and
\begin{align*}
L = &\sum\limits_{\substack{2\leq p_2\leq m+1 \\ 0\leq q_1 \leq p_2-2 \\ 0 \leq q_2 \leq m+1-p_2}}\omega_2(L_{1,p_2,q_1,q_2}^{m+1}) + \sum\limits_{\substack{2 \leq p_1 \leq m-1 \\ 0 \leq q_1 \leq m-p_1-1}}\omega_2(L_{p_1,m,q_1,m+1-p_2}^{m+1}) \\
 &\quad+ \sum\limits_{\substack{2 \leq p_1 \leq m \\ 0 \leq q_1 \leq m-p_1}}\omega_2(L_{p_1,m+1,q_1,0}^{m+1}) + \sum\limits_{\substack{2 \leq p_1 \leq m \\ p_1+1 \leq p_2 \leq m-1 \\ 0 \leq q_1 \leq p_2-p_1-1}} \omega_2(L_{p_1,p_2,q_1,m+1-p_2}^{m+1})
\end{align*}

\subsection{Proof that $\sigma = L$}

We show that in fact the values of $\sigma$ and $L$ above are equal. We will show that there is a bijection of sets $f:\sigma \to L$ that preserves the product of the weights.

\begin{proof}
Let $\delta$ be a disjoint $(m+1)$-collection in $\sigma$. In general, $\delta$ will have its first flat step at $p_1$ and second flat step at $p_2$. In between the two flat steps there will be $q_1$ M-paths followed by $p_2-p_1-q_1$ number of V-paths. We'll call the height at which this first V-path appears to be $v$. Then we define $f(\delta)$ to be a disjoint $(m+1)$-collection in $L$ where we take $\delta$ and replace the V-path at $v$ with a off-centred flat step path and all the paths from $v+1$ up to $p_2$ are replaced with asymmetric M-paths. If $\delta$ had no V-paths in between the first two flat steps, then just replace the second flat step with an off-centred flat step. Clearly $f(\delta)$ is a unique disjoint $(m+1)$-collection in $L$ and this defines a well-defined function of sets from $\sigma$ to $L$. We can see that $f$ preserves the product of the weights on $\delta$: each V-path starting at a height of say, $h$ had product of weights $\frac{(2h-1)!}{(h-1)!}$ and this path was replaced by an asymmetric M-path with product of weights $\frac{(2h-2)!}{(h-1)!}$ but on the asymmetric M-path of height $h+1$ we have an extra factor of $(2(h+1)-3) = (2h+2-3) = (2h-1)$ so in total we also have product of weights $\frac{(2h-1)!}{(h-1)!}$. Note that this also applies to the flat step we introduced at height $v$: it had product of weights $\frac{(2v-2)!}{(v-1)!}$ but the asymmetric M-path just above it gives an extra factor of $(2v-1)$ which is already accounted for. Finally the flat step at $p_2$ in $\delta$ had product of weights $\frac{(2p_2-2)!}{(p_2-1)}$ which is the same as the product of the weights in the asymmetric M-path we introduced at $p_2$. So we see that $f$ preserves weighting.

Now we define a function $g:L\to\sigma$. Let $\delta$ instead be a disjoint $(m+1)$-collection in $L$. The collection $\delta$ has a second off-centred flat step at height $p_2$ with $q_1$ number of asymmetric M-paths above it. Then we define $g(\delta)$ to be the disjoint $(m+1)$-collection where we replace the second flat step at $p_2$ and all the asymmetric M-paths above it with V-paths except for the last one, which we turn into a centred flat step (ie. at height $p_2+q_1$. The paths above $p_2+q_1$ will be either V-paths or (symmetric) M-paths and so this gives us a disjoint $(m+1)$-collection in $\sigma$. Clearly, $g,f$ are inverse to each other, giving us a bijection $\sigma\to L$. Since $f$ preserves products of weights, this also gives us that $\sigma = L$ as values.
\end{proof}

\subsection{$\sigma$ Term}

Since we understand V-paths, we can immediately write down an explicit expression for $N$:
\begin{equation*}
N = \prod\limits_{\tau\in\sigma_{\text{roof}}^{m+1}}\omega_2(\tau) = \prod\limits_{k=1}^{m+1}\frac{(2k-1)!}{(k-1)!}
\end{equation*}

So by above, we have
\begin{equation*}
\frac{N''-d!D''}{N} = \frac{3\sigma}{N}
\end{equation*}

Earlier we had split $\sigma$ into four smaller sums based on which case they fell into. Consider a disjoint $(m+1)$-collection in the first case, that is, we fix $p_1 = 1$, $2\leq p_2\leq m+1$, $0\leq q_1\leq p_2 - 2$, $0\leq q_2 \leq m+1-p_2$. Then we have
\begin{align*}
\frac{\sigma'}{N} &= \frac{\left(\prod\limits_{k=2}^{q_1+2}(2k-2)!(2k-2)\right)\left(\prod\limits_{k=q_1+2}^{p_2-1}(2k-1)!\right)(2p_2-2)!\left(\prod\limits_{k=p_2+1}^{p_2+q_2}(2k-2)!(2k-2)\right)\left(\prod\limits_{k=p_2+q_2+1}^{m+1}(2k-1)!\right)}{\left(\prod\limits_{k=0}^{m+1}(2k-1)!\right)} \\
&= \frac{\left(\prod\limits_{k=2}^{q_1+2}(2k-2)!(2k-2)\right)(2p_2-2)!\left(\prod\limits_{k=p_2+1}^{p_2+q_2}(2k-2)!(2k-2)\right)}{\left(\prod\limits_{k=2}^{q_1+2}(2k-1)!\right)\left(\prod\limits_{k=p_2}^{p_2+q_2}(2k-1)!\right)} \\
&= \frac{1}{2p_2-1}\left(\prod\limits_{k=2}^{q_1+2}\frac{(2k-2)}{(2k-1)}\right)\left(\prod\limits_{k=p_2+1}^{p_2+q_2}\frac{(2k-2)}{(2k-1)}\right)
\end{align*}

So for the first case, we have
\begin{equation*}
\frac{\sigma_1}{N} = \sum\limits_{\substack{2\leq p_2\leq m+1 \\ 0\leq q_1 \leq p_2-2 \\ 0\leq q_2 \leq m+1-p_2}}\frac{1}{2p_2-1}\left(\prod\limits_{k=2}^{q_1+2}\frac{(2k-2)}{(2k-1)}\right)\left(\prod\limits_{k=p_2+1}^{p_2+q_2}\frac{(2k-2)}{(2k-1)}\right)
\end{equation*}

We similarly find expressions for the other three smaller sums.

Consider a disjoint $(m+1)$-collection in the second case, that is, we fix $2\leq p_1\leq m-1$,$p_2=m$,$0\leq q_1\leq m-p_1-1$, $q_2 = 1$. Then we have
\begin{equation*}
\frac{\sigma'}{N} = \frac{1}{2p_1-1}\left(\prod\limits_{k=p_1+1}^{p_1+q_1}\frac{(2k-2)}{(2k-1)}\right)\left(\frac{2m}{(2m-1)(2m+1)}\right)
\end{equation*}
and for the whole second case, we have
\begin{equation*}
\frac{\sigma_2}{N} = \sum\limits_{\substack{2\leq p_1\leq m-1 \\ 0\leq q_1\leq m-p_1-1}}\frac{1}{2p_1-1}\left(\prod\limits_{k=p_1+1}^{p_1+q_1}\frac{(2k-2)}{(2k-1)}\right)\left(\frac{2m}{(2m-1)(2m+1)}\right)
\end{equation*}

Consider a disjoint $(m+1)$-collection in the third case. Then we have
\begin{equation*}
\frac{\sigma'}{N} = \frac{1}{2p_1-1}\left(\prod\limits_{k=p_1+1}^{p_1+q_1}\frac{(2k-2)}{(2k-1)}\right)\left(\frac{1}{2m+1}\right)
\end{equation*}
and for the whole third case, we have
\begin{equation*}
\frac{\sigma_3}{N} = \sum\limits_{\substack{2\leq p_1\leq m \\ 0\leq q_1\leq m-p_1}}\frac{1}{2p_1-1}\left(\prod\limits_{k=p_1+1}^{p_1+q_1}\frac{(2k-2)}{(2k-1)}\right)\left(\frac{1}{2m+1}\right)
\end{equation*}

Consider a disjoint $(m+1)$-collection in the fourth case. Then we have
\begin{equation*}
\frac{\sigma'}{N} = \frac{1}{2p_1-1}\left(\prod\limits_{k=p_1+1}^{p_1+q_1}\frac{(2k-2)}{(2k-1)}\right)\frac{1}{2p_2-1}\left(\prod\limits_{k=p_2+1}^{m+1}\frac{(2k-2)}{(2k-1)}\right)
\end{equation*}
and for the whole fourth case, we have
\begin{equation*}
\frac{\sigma_4}{N} = \sum\limits_{\substack{2\leq p_1\leq m \\ p_1+1\leq p_2 \leq m-1 \\ 0\leq q_1\leq p_2-p_1-1 }}\frac{1}{2p_1-1}\left(\prod\limits_{k=p_1+1}^{p_1+q_1}\frac{(2k-2)}{(2k-1)}\right)\frac{1}{2p_2-1}\left(\prod\limits_{k=p_2+1}^{m+1}\frac{(2k-2)}{(2k-1)}\right)
\end{equation*}

\subsection{Putting It All Together}

So all together we have
\begin{align*}
&\frac{d^2}{dt^2}\text{Mag}(tB_2^d)\big\vert_{t=0} = \\
&6\sum\limits_{\substack{2\leq p_2\leq m+1 \\ 0\leq q_1 \leq p_2-2 \\ 0\leq q_2 \leq m+1-p_2}}\frac{1}{2p_2-1}\left(\prod\limits_{k=2}^{q_1+2}\frac{(2k-2)}{(2k-1)}\right)\left(\prod\limits_{k=p_2+1}^{p_2+q_2}\frac{(2k-2)}{(2k-1)}\right) + \\
&6\sum\limits_{\substack{2\leq p_1\leq m-1 \\ 0\leq q_1\leq m-p_1-1}}\frac{1}{2p_1-1}\left(\prod\limits_{k=p_1+1}^{p_1+q_1}\frac{(2k-2)}{(2k-1)}\right)\left(\frac{2m}{(2m-1)(2m+1)}\right) + \\
&6\sum\limits_{\substack{2\leq p_1\leq m \\ 0\leq q_1\leq m-p_1}}\frac{1}{2p_1-1}\left(\prod\limits_{k=p_1+1}^{p_1+q_1}\frac{(2k-2)}{(2k-1)}\right)\left(\frac{1}{2m+1}\right) + \\
&6\sum\limits_{\substack{2\leq p_1\leq m \\ p_1+1\leq p_2 \leq m-1 \\ 0\leq q_1\leq p_2-p_1-1 }}\frac{1}{2p_1-1}\left(\prod\limits_{k=p_1+1}^{p_1+q_1}\frac{(2k-2)}{(2k-1)}\right)\frac{1}{2p_2-1}\left(\prod\limits_{k=p_2+1}^{m+1}\frac{(2k-2)}{(2k-1)}\right) - \\
&V_{1}\sum\limits_{\substack{1\leq p \leq m-1 \\ 0 \leq q \leq m - 1 - p}}\frac{1}{2p+1}\prod\limits_{k=p+1}^{p+q}\left(\frac{2k}{2k+1}\right)
\end{align*}

\subsection{Skip Factorials}

The products
\begin{equation*}
\prod\limits_{k=a}^{b}\frac{(2k-2)}{(2k-1)},\qquad\prod\limits_{k=a}^b\frac{2k}{2k+1}
\end{equation*}
that appear in the sums above are ratios of skip or double factorials. In the following, we will refer to the identities about skip factorials given just below:
\begin{align*}
&(2n)!! = 2^nn! \\
&(2n-1)!! = \frac{(2n)!}{2^nn!} \\
&(2n+1)!! = \frac{(2n+1)!}{2^nn!}
\end{align*}
We will also use Catalan numbers, which are defined by
\begin{equation*}
C_n = \frac{1}{n+1}\binom{(2n)}{n} = \frac{1}{2n+1}\binom{(2n+1)}{n}
\end{equation*}
Now let's write the product
\begin{equation*}
\prod\limits_{k=a}^{b}\frac{(2k-2)}{(2k-1)}
\end{equation*}
in terms of skip factorials and try to simplify from there:
\begin{align*}
\prod\limits_{k=a}^{b}\frac{(2k-2)}{(2k-1)} &= \frac{2(a-1)}{2a-1}\cdots\frac{2(b-1)}{2b-1} \\
&= \frac{(2(b-1))!!}{(2(a-2))!!}\left[\frac{(2b-1)!!}{(2(a-1)-1)!!}\right]^{-1} \\
&= \frac{2^{b-1}(b-1)!}{2^{a-2}(a-2)!}\left[\frac{(2b)!}{2^bb!}\frac{2^{a-1}(a-1)!}{(2(a-1))!}\right]^{-1} \\
&= \frac{2^{2b-1}}{2^{2(a-1)-1}}\frac{b!(b-1)!}{(2b)!}\frac{(2(a-1))!}{(a-1)!(a-2)!} \\
&= \frac{2^{2b-1}}{2^{2(a-1)-1}}\frac{(a-1)\binom{2(a-1)}{a-1}}{b\binom{2b}{b}} \\
&= \frac{2^{2b-1}}{2^{2(a-1)-1}}\frac{(a-1)aC_{a-1}}{b(b+1)C_b}
\end{align*}
And similarly:
\begin{align*}
\prod\limits_{k=a}^b\frac{2k}{2k+1} &= \frac{2a}{2a+1}\cdots\frac{2b}{2b+1} \\
&= \frac{(2b)!!}{(2(a-1))!!}\left[\frac{(2b+1)!!}{(2a-1)!!}\right]^{-1} \\
&= \frac{2^bb!}{2^{a-1}(a-1)!}\left[\frac{(2b+1)!}{2^bb!}\frac{2^aa!}{(2a)!}\right]^{-1} \\
&= \frac{2^{2b}}{2^{2a-1}}\frac{(b!)^2}{(2b+1)!}\frac{(2a)!}{a!(a-1)!} \\
&= \frac{2^{2b}}{2^{2a-1}}\frac{a\binom{2a}{a}}{(b+1)\binom{2b+1}{b}} \\
&= \frac{2^{2b}}{2^{2a-1}}\frac{a(a+1)C_a}{(2b+1)(b+1)C_b}
\end{align*}

Using the above, we can rewrite the last sum in our expression for the second derivative as the following:
\begin{equation*}
V_{1}\sum\limits_{\substack{1\leq p \leq m-1 \\ 0 \leq q \leq m - 1 - p}}\frac{1}{2p+1}\prod\limits_{k=p+1}^{p+q}\left(\frac{2k}{2k+1}\right) = V_1\sum\limits_{\substack{1\leq p \leq m-1 \\ 0 \leq q \leq m-1}} \frac{1}{2p+1}\frac{2^{2(p+q)}}{2^{2(p+1)-1}}\frac{(p+1)(p+2)C_{p+1}}{(2(p+q)+1)(p+q+1)C_{p+q}}
\end{equation*}
Setting $k = p+q$, this last sum turns into
\begin{align*}
&V_1\sum\limits_{k=1}^{m-1}\sum\limits_{p=1}^k\frac{1}{2p+1}\frac{2^{2k}}{2^{2(p+1)-1}}\frac{(p+1)(p+2)C_{p+1}}{(2k+1)(k+1)C_k} = V_1\sum\limits_{k=1}^{m-1}\frac{2^{2k}}{(2k+1)(k+1)C_k}\sum\limits_{p=1}^k\frac{(p+1)(p+2)C_{p+1}}{2^{2p+1}(2p+1)}
\end{align*}

\nocite{*}
\bibliographystyle{alpha}
\bibliography{refs.bib}

\end{document}